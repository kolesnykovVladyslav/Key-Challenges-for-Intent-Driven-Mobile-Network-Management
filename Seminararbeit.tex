\input{preamble}

\usepackage{xcolor} 
\usepackage{listings}
\lstdefinelanguage{Nile}{
    morekeywords={define, intent, from, to, for, add, with, allow, start, end},
    sensitive=false, % keywords are not case-sensitive
    morecomment=[l]{//}, % l is for line comment
    morecomment=[s]{/*}{*/}, % s is for start and end delimiter
    morestring=[b]", % defines that strings are enclosed in double quotes
    showstringspaces=false,
    keywordstyle=\color{blue},
    commentstyle=\color{green},
    backgroundcolor=\color{white},
    stringstyle=\color{violet},
    numbers=none,
    frame=none
} %


\begin{document}
\pagenumbering{roman}	
\begin{titlepage}
  \onehalfspacing
  \makeatletter
  \vspace*{1em}
  \begin{center}
    \ifdefempty{\@subject}{}{%
      {\usekomafont{subject}\@subject}
      \par\vspace{2em}
    }
    {\usekomafont{title}\@title}
    \ifdefempty{\@subtitle}{}{%
      \par\vspace{.5em}
      {\usekomafont{subtitle}\@subtitle}
    }
    \par\vspace{2em}
    \singlespacing
    {\usekomafont{author}%
      \@author\par
      \GetTranslation{matrno}: \@matrno\par}
    \texttt{\@email}
    \par\vspace{1.5em}
    {\usekomafont{publishers}\@publishers}
  \end{center}
  \makeatother

  \begin{abstract}
    \noindent%
    \paragraph*{\abstractname}
    
    Intent-Driven Networking (IDN) is a promising concept that tends to automate conventional networking by applying a deeper level of intelligence. High-level intents offer an efficient and powerful way of expressing network requirements to manage its infrastructure, resources, and services without providing exact configurations. This paper surveys the main challenges of applying IDN in mobile networks. We clarify the definition of IDN, present a generic architecture and briefly cover an intent translation process. We also discuss the current research status of IDN and present its clear advantages over regular networking.
    
  \end{abstract}

  \vfill
  \centering
  \includegraphics[height=38mm]{figures/uni_siegel}
\end{titlepage}
%%% Local Variables:
%%% mode: latex
%%% TeX-master: "Seminararbeit"
%%% End:


\tableofcontents

\clearpage
\pagenumbering{arabic}

\section{Introduction}
\label{sec:Introduction}
Intent-Driven Networking (IDN) is a promising network concept that has received great attention from industry and open source communities in recent years. Conventional networks are configured manually with specific execution commands. Instead, the network is provided with desired business intent and does not require any exact configuration. This concept has the potential to automate the network management process by providing simple and powerful tools to handle resources, infrastructure, and services\cite{Mwanje2021}. However, there is no unified and clear definition of Intent-Driven Networking. The enabling techniques and frameworks of IDN are still in the further exploration phase\cite{8968429}.

\section{Basics}
\label{sec:Basics}

\subsection{Mobile Networks}
TODO: 5G and 6G Mobile Networks


\subsection{Software-Defined Networking and Network Function Virtualization}

Virtualization and softwarisation enable high-level abstraction of network management functions. These concepts are key enablers of IDN.

\textbf{Software-Defined Networking (SDN)} is a concept that enables a dynamic and programmatically efficient network configuration in order to improve network performance and flexibility. The data plane, which is a layer where devices are interconnected through wireless radio channels or wired cables, is decoupled from the control plane and managed by a centralized controller\cite{Li2015}. The SDN controller logically maintains the dynamic network, takes requests from the application layer, and manages the network devices via standard protocols. Simply put, the network is programmable through software applications running on top of the SDN controller that interacts with the underlying devices. It allows operators to manage the network and services from a centralized user interface. In \cite{Amani2014}, SDN was also proposed for the mobile networks to efficiently accommodate the explosive data traffic over the LTE or Wi-Fi accesses, considering the real-time network condition.

\textbf{Network Function Virtualization (NFV)} is a complementary technology to the SDN, which can provide the infrastructure on which SDN can run. Network virtualization is an approach that decouples the network functions from their supported hardware and allows them to be implemented as software. NFV makes it possible to programmatically create, provision, and manage networks in software while using the underlying physical network. Virtualization decouples network services, such as routers, firewalls, and load balances, from the specific hardware and allows virtual provisioning of an entire network. The combination of SDN and NFV can address the challenges of dynamic resource management and intelligent service orchestration.\cite{Li2015}

IDN takes traditional networking management one step further. It builds upon SDN and NFV technologies enabling a higher level of intelligence and automation in networks. IDN allows configuring a network to address the desired business goal by specifying just what the resulting state of the system shall be instead of how the system has to reach it.


\section{Intent-Driven Networking}
\label{sec:Intent_Driven_Networking}
The next section presents the formal definitions of Intent and IDN. 

\subsection{Definition of Intent-Driven Networking}
Though different research institutions have proposed various definitions of intent and IDN, all presented interpretations closely resemble each other. 

The intent can be described as a declarative way to define the desired state of the system that can be translated into an advanced policy. It abstracts the components and capabilities of the network from a requirements perspective. The intent does not define any specific commands on how to reach the resulting state. Instead, it only specifies the desired outcome of the network. For example, a cell should have a range of 1 km with a data rate of 100 Mbit/s.\cite{Mwanje2021} It means that the network should be able to understand the intent and map it into a specific device-level configurations.  The system automatically observes its current state and adjusts itself to achieve the desired outcome if there are any deviations. \cite[22867]{8968429}

Intent-Driven Networking is a technology concept that tends to automate network administrative tasks by applying a deeper level of intelligence and intended state. IDN captures, translates, and applies business intent across the network. The network is able to automatically deploy, configure, verify and optimize itself to achieve a target state defined by intent. The network performs continuous monitoring and adjustment to ensure alignment, which is achieved through a closed-loop system. Eventually, IDN provides full-lifecycle management for the network.\cite{8968429}

IDN management provides four main functionalities\cite[271]{Wei2020}: 

\begin{enumerate}
	\item Translation and verification. The IDN captures and translates business intent into the necessary system configuration, which is then automatically applied across the network. The system then validates the correctness of the resulting policy.
	\item Automatic deployment. The IDN installs translated and verified policies across physical and virtual network infrastructure through network automation and network orchestration.
	\item Network state awareness. The state of the network may constantly change, which can lead to some inconsistencies between implementation and the desired outcome. The system needs to continuously monitor and manage the network to guarantee fulfillment of the business intent at any point in time.
	\item Accurate diagnosis and dynamic optimization. The ultimate goal is for the network to continuously monitor and adjust network performance to ensure the desired business outcome. If expected intent is not achieved, the system takes corrective actions.
\end{enumerate}


\subsection{Intent-Driven System Architecture}

Different research groups have various concepts and design solutions for the IDN architecture. Figure \ref{fig:IDN_Architecture} illustrates a generalized architecture for IDNs presented in \cite[3]{Saha2018} that was inspired by the Open Platform for Network Function Virtualization (OPNFV) project. Its core elements are discussed below.  

\begin{figure}[htb]
  \centering
  \includegraphics[width=0.8\textwidth]{figures/IBN_Architecture.png}
  \caption{Basic Architecture of IDN\cite{Saha2018}}
  \label{fig:IDN_Architecture}
\end{figure}

\textbf{Intent Processor (IP)} is the core component of IDN architecture that accepts high-level intent provided by users or various applications through a corresponding interface. Intents can be given in different formats, such as simple text, executable commands via a Command-Line Interface (CLI), interaction with a graphical user interface (GUI), or audio interface as speech. After new intent is provided, the translater module performs mapping into a machine-readable representation for the required modules or services. For instance, IP can translate plain text into some internal representation (e.g. Yet Another Next Generation (YANG) as a possible modeling language) with the help of various Natural Language Processing (NLP) technologies\cite[15]{Mehmood2021}. The IP is also responsible for intent assurance. Based on the information from State Observer, IP takes corrective action and assists in the reconfiguration of the underlying infrastructure to preserve the required intents.

\textbf{Optimization Decision Maker (ODM)} is responsible for intent verification optimization and decision making. After receiving as input a mid-level representation of intent, ODM tries to find a feasible and the most optimal solution that is then returned to IP. Based on AI- and ML techniques, the AI Engine (AIE) of ODM learns from past intents to make better decisions in terms of feasibility and intent installation. In addition, this module may be trained on rewards and penalties provided by IP.

\textbf{Data Store (DS)} may store all relevant data collected from the network in a centralized or distributed manner. It provides an API so that other components can store and retrieve data from the DS. Additionally, DS may have a Graph Builder (GB) module that provides a graph view of the underlying network. The graph model can be utilized for optimization, decision making, generating analytics, and network visualization.

\textbf{State Observer (SO)} is a crucial component for intent assurance because it is responsible for monitoring the network. Based on data retrieved from DS it observes the current state and informs IP if there is any threat or inconsistency against installed intent. For instance, a node can be overloaded with network traffic leading to failure in serving a critical service. In this case, the SO identifies the threat and informs the Intent Processor.

The following modules, such as \textbf{Visualizer} and \textbf{Analytics Engine (AE)} are optional components of IDN but still provide some useful functionality. The Visualizer displays information about the state and performance of the network at a glance using a graphical user interface (GUI). It ensures total visibility of the entire network by providing graphical representations of network devices, network metrics, and data flows. Analytics Engine is used to process queries related to network performance, based on raw monitoring data and graph model. Third-party applications may access AE using its API to perform network monitoring.

In IDN, \textbf{applications} are transparent to the underlying system and only interact via appropriate Application Programming Interfaces (APIs) or so-called Northbound Interface (NBI). These APIs play a crucial role as they hide the underlying system's complexities.

The underlying resource layer provides the physical network infrastructure with various device entities. It contains multiple controllers like \textbf{Network (NC)}, \textbf{Storage (SC)}, and \textbf{Computation Controller (CC)} that configure the network devices, such as routers, switches, disks, and servers, according to a required intent. The NC can be an SDN controller or network management system in a traditional network. The \textbf{Infrastructure Manager (IM)} represents an interface between the IP and the different controllers. IM translates feasible intents into domain-specific instructions and also collects various statistics from the controllers.

In comparison, the authors in \cite[13]{Mehmood2021} present another IDN architecture, which is more detailed and contains four following layers: application, intent, network management and orchestration, and resource layers. This survey gives a comprehensive overview of the functions and features of each layer with its key enabling technologies.


\subsection{Intent Translation}

The paper \cite{Jacobs2018} introduces an intent-refinement process that uses machine learning and feedback from the operator to translate intents into network configurations. The proposed refinement method extracts an intermediate representation of intent from natural language before compiling it into SDN commands.

The intent is collected through a chat interface, and a neural network learning model is used to extract entities from natural language. Examples of such entities are the network endpoints, middleboxes, and temporal configurations for the policy. A previously trained sequence-to-sequence learning model based on a Recurrent Neural Network (RNN) with Long Short-Term Memory is used to translate extracted entities to structured intents written in the Nile language. 

Using the Nile language, the operator can build powerful yet simple intents. For example, an input \textit{"Add firewall and intrusion detection from gateway to backend for client B, with latency less than 10ms and 100mbps of bandwidth, and allow HTTPS only, every day from 09:00 to 18:00"} can be represented as in \Cref{lst:nile_intent_example}. The extracted Nile intent acts as an abstraction layer for lower-level configuration and policy languages. Authors claim the Nile's grammar is expressive enough to represent most network intents.


\begin{lstlisting}[language=Nile,float=ht,caption={Nile intent example\cite{Jacobs2018}},label=lst:nile_intent_example]
define intent qosIntent:
  from    endpoint("gateway")
  to      endpoint ("database")
  for     client("B")
  add     middlebox("firewall"), middlebox("ids")
  with    latency("less", "10s"),
          throughput("more or equal", "100mbps")
  allow   traffic ("https")
  start   hour("09:00")
  end     hour("18:00")
\end{lstlisting}

The structured intent definition generated by the translation model is then presented to the network operator for confirmation. The operator may either confirm the correctness of the intent program or make adjustments if necessary. Based on the user’s response, the translation model is then trained, ensuring that the results improve every time the operator provides new intent.

Finally, having a structured intermediate intent representation verified by the user, the Intent Deployer can compile and deploy it into configuration commands using SONATA-NFV, which is an emulation platform that deploys network functions as Docker containers. Decoupling provided by intermediate intent representation ensures the reusability of the proposed solution by allowing compilation of the intents to other existing network configurations and policy languages, such as Janus, PGA, and Kinectic.

However, intent translation is not limited to this implementation. The same paper \cite[20]{Jacobs2018} also briefly covers other existing intent languages, frameworks, and compilers, discussing their goals and shortcomings.
	

\subsection{Closed-loop System}

\subsection{Intent Fulfilment}

\subsection{Industrial Products}



\subsection{Advantages of Intent-Driven Networking}

Intent-Driven Networking has clear advantages over regular networking. Most of these advantages lead to significant time savings and improve the agility of the entire system\cite{Kolibri}. The main benefits of IDNs are listed below:\cite[11]{MartinezJulia2022}

\textbf{Simplifies and automates network operations.} IDN reduces the complexity of the management and maintenance of the entire infrastructure. It facilitates work associated with the traditional network configuration and simplifies the deployment of additional network services.

\textbf{Automates troubleshooting and resolution.} Through its closed-loop design, an intent-based network reduces complex troubleshooting scenarios. As assurance processes verify the alignment of network configuration with intent, IDN can recognize potential problems before they occur and immediately perform correcting actions.

\textbf{Enables automatic learning and optimization.} Using machine learning techniques, IDN may predict any violations of the expressed intent before changes will be applied, forecast trends, identify anomalies, predict and validate system-level performance. In addition, IDN may learn from past decisions to determine the most optimal solution.

\textbf{Improves network security capabilities.} Part of the monitoring is that IDN is always looking for threats in the network data. Security violations are immediately identified and prevented.

\textbf{Continuous alignment of the network with the business objectives.} IDN ensures that the network always stays in compliance with the business goals by constantly performing intent assurance and implementation.

\textbf{Fast implementation of business goals into network configurations.} Users can quickly provide high-level business intents, which are then translated into optimal network policies. The configuration of the network components is fully automated.



\subsection{Challenges of Intent-Driven Networking}

IDN improves the automation of networks and abstracts complex problems, but it also faces some challenges. This section covers several issues that require further research.

\subsubsection{Intent description and translation}
In IDNs, it is expected for the intents to be human-readable at the application level and more machine understandable as the intent flows through the whole system towards the resource layer. The machine-readable format as a mid-level representation is required for efficient intent processing and implementation. However, a human-readable format is preferable for users since arbitrary complex goals can be expressed in a high-level and flexible way. The translation of abstract intents into concrete network configurations is one of the biggest challenge and design goals in IDN systems.

Several different approaches have been proposed for intent representation, such as ontology-based, unrestricted language, vocabulary-restricted, and other hybrid methods. The intent translation is aided by various Natural Language Processing (NLP) technologies. The capabilities of these approaches are still limited compared to naturally spoken language. The intent representation and translation problems still have to be explored to provide scalable solutions for human-readable intent description models.\cite[18]{Mehmood2021} 

\subsubsection{Lack of standardization}
Different research groups, open-source communities, and industrial companies independently explore and develop IDN solutions. Existing IDN frameworks are based on heterogeneous architectures and underlying techniques that present different functional characteristics. Each component and architectural layer heavily relies on API, and the current unified standard has not been formed. Thus, the absence of a standardized format for specifying intents and common architecture is one of the main challenges of IDN.\cite{8968429}

\subsubsection{Application of Artificial Intelligence}
The increasing complexity of the mobile networks requires IDNs to be able to translate users' business intents and other requirements into the network configuration, operation, and maintenance policies, and perform the self-configuration, self-optimization, and self-healing of the network. These tasks can be accomplished by AI technologies to ensure network performance while reducing operational costs and improving network robustness. Thus, mobile networks would greatly benefit from advances in the domains of Artificial Intelligence and Machine Learning.\cite{Wei2020}

They are some potential areas that require the integration of AI in the IDN framework:
\begin{itemize}
\item[--] Intent translation.
\item[--] Intent and service assurance with proactive monitoring of resources and network environment.
\item[--] Optimization of the network and service orchestration.
\item[--] Intent-to-service mappings.
\item[--] Troubleshooting and resolution.
\item[--] Security assurance.
\end{itemize}

For instance, NLP technologies can be used for understanding the intent structure and its translation through Sequence-to-Sequence (seq2seq) learning models such as recurrent neural networks (RNN, Long-Short-Term Memory (LSTM), Gated Recurrent Unit (GRU)) and advanced models including attention, Bidirectional Encoder Representations from Transformers (BERT). The context and declarative parameter retrieval leading to the appropriate network configuration can be achieved through a Convolutional Neural Network (CNN).\cite{Mehmood2021}

Network monitoring could be challenging in new generation mobile networks due to the dynamic and heterogeneous nature of the cellular environment. With AI-powered diagnostic analytics, the network can quickly and accurately detect problems and resolve them, even before they occur. Based on data from various sources, algorithms like Logistic Regression (LR), Support Vector Machine (SVM), and Hidden Markov Model (HMM) could be utilized for network anomaly detection.\cite{anuradha2017empowering}

The research on the integration of IDNs and AI approaches into mobile networks has just started in recent years. The current studies focus mostly on the core networks, and intent-driven mobile networks need further exploration before IDN makes a breakthrough in wireless communication technology, achieving a novel standard in mobile networks.

\subsubsection{Legacy networks}
Industries and enterprises tend to move away from legacy network technologies as more efficient solutions become available and the nature of consumer demand changes. However, the legacy network equipment and systems are still often in use. The aspect of migrating them to IDN architecture has to be considered.\cite{Saha2018}



\section{Conclusion}
\label{sec:Conclusion}





\section{TODO: Delete}
\label{sec:Delete}
Abbildungen können z.\,B. im Unterverzeichnis \texttt{figures} abgelegt werden.
Eingebunden werden diese mit dem Befehl \texttt{\textbackslash includegraphics} innerhalb
einer \texttt{figure}-Umgebung:
\begin{figure}[htb]
  \centering
  \includegraphics[width=0.8\textwidth]{figures/Hummingbird.jpg}
  \caption{Eine Veilchenkopfelfe (auch Costakolibri genannt, vom lateinischen Calypte costae), die zur Familie der Kolibris gehört }
  \label{fig:kolibri}
\end{figure}

\subsection{Erste Zwischenüberschrift}
\label{sec:ErsteZwischenueberschrift}
Die Arbeit kann auch Tabellen im \texttt{table}-Environment enthalten:
\begin{table}[ht]
  \centering
  \caption{Entfernungstabelle Süddeutschland, vgl. \cite{entfernungstabelle}}
  \begin{tabular}{c r r r}
    \toprule
              & Augsburg & München & Stuttgart \\
    \midrule
    Augsburg  & -        & 61      & 149       \\
    München   & 61       & -       & 210       \\
    Stuttgart & 149      & 210     & -         \\
    \bottomrule
  \end{tabular}
  \label{tab:entfernungen}
\end{table}

\subsubsection{Erste Unterüberschrift}
\label{sec:ErsteUnterueberschrift}

Das \texttt{listings}-Paket erlaubt es, Quellcode mit Syntax-Highlighting einzubinden:

\begin{lstlisting}[language=Python,float=ht,caption={Python-Programm zur Berechnung der Fakultätsfunktion},label=lst:factorial]
def fact(n):
    """Return the n-th factorial number"""
    if n == 0:
        return 1
    else:
        return n * fact(n-1)
  
# Test output
print fact(10)
print "Done"
\end{lstlisting}

In \Cref{lst:factorial} finden Sie eine rekursive Funktion zur Fakultätsberechnung.
  
\subsection{Zweite Zwischenüberschrift}
\label{sec:ZweiteZwischenueberschrift}

TEXT

% Literaturverzeichnis
\printbibliography[heading=bibintoc]

% Anhang
\include{appendix}

% Eidesstattliche Erklärung
\clearpage
\section*{\GetTranslation{honesty@title}}
\GetTranslation{honesty@body}

\vspace{2em}
\makeatletter
Augsburg, \@date
\par\vspace{1.5cm}
(\@author)
\makeatother

%%% Local Variables:
%%% mode: latex
%%% TeX-master: "Seminararbeit"
%%% End:


\end{document}