\documentclass[a4paper,DIV=16]{scrartcl}

\usepackage[utf8]{inputenc}
\usepackage[T1]{fontenc}
\usepackage[german]{babel}

\usepackage[sfdefault]{FiraSans}
\usepackage{FiraMono}
\usepackage{url}
\usepackage{xcolor}
\definecolor{smdsblue}{RGB}{0, 69, 134}
\usepackage[color=smdsblue!25]{todonotes}
\usepackage{local_style}

\pagestyle{empty}

\begin{document}
\title{Review Seminararbeit}
\date{SoSe 2020}

%% >> Bitte entsprechend Ihres EIGENEN Seminars ausfüllen!
%% --- 8< --- 8< --- 8< ---
%\subtitle{Seminar Automotive Software and Systems Engineering}
%\subtitle{Seminar Medical Information Science}
\subtitle{Seminar Software Engineering für verteilte Systeme}
%\subtitle{Seminar Software Engineering in sicherheitskritischen Systemen}
%% --- >8 --- >8 --- >8 ---

% >> Bitte "\personal{...}" stehen lassen und nur den Inhalt "\todo[......" mit ihrem Namen ersetzen. 
% >> Dieses Kommando wird später zur Anonymisierung genutzt. 
\author{\personal{Vladyslav Kolesnykov (2054429)}}

\maketitle
\thispagestyle{empty}

\section*{Hinweise}
\begin{itemize}
\item Länge des Reviews: 2--3 Seiten (inklusive dieser Hinweise)
\item Jede Frage in diesem Template muss beantwortet werden! \textbf{Ersetzen} Sie dazu die vorhandenen \texttt{\textbackslash{}todo}-Befehle im Template durch Ihre Antworten in Fließtext oder ausführlichen Stichpunkten.
\item Die Qualität der von Ihnen verfassten Reviews geht in Ihre Gesamtnote für das Seminar ein.
\item Entfernen Sie nach dem vollständigen Bearbeiten des Reviews das \texttt{todonotes}-Paket im Header dieser Datei. Wenn Sie alle \texttt{\textbackslash{}todo}-Befehle ersetzt haben, kompiliert das Dokument weiterhin ohne Fehler.  \texttt{\textbackslash{}usepackage[color=smdsblue!25]{todonotes}}
\end{itemize}

\section*{Allgemeine Informationen}
\subsection*{Titel der zu bewertenden Arbeit}

Bewertung der Quality of Experience von Videostreaming auf Basis von Netzwerkverkehr


\section*{Hauptinhalt der Arbeit}

This paper provides an overview of existing approaches and metrics for the evaluation of the quality of experience in the video streaming domain, discussing their application possibilities and limitations. The introduction presents a motivating example that highlights the importance to evaluate the quality of experience and service while reducing the amount of traffic by utilizing passive monitoring. End-to-end encryption of data represents an additional challenge.

Section 2 covers various metrics for measuring quality in the video streaming domain. There are two major groups: Quality of Service (QoS) and Quality of Experience (QoE) metrics. The QoS presents the overall performance of a streaming service or its components and includes the following metrics: availability, throughput, transmission delay, and jitter. In contrast, the Quality of Experience (QoE) examines a user's satisfaction with a system in use. Direct and indirect QoE metrics have emerged for the quality analysis of internet video streaming. The direct metrics include all factors that directly affect the quality of media reproduction as perceived by a user. The following direct metrics were presented: Peak-Signal-to-Noise-Ratio, Structural Similarity, Video Quality, and Mean Opinion Score. While, indirect metrics are not directly related to the quality of content, but still influence media delivery. Such metrics are start-up time, system response time, delivery synchronization, freshness, and buffering. However, due to the difficulty of distinguishing the QoE and QoS metrics, the term Quality of Performance (QoP) is used to define both groups.

The main part provides an overview of existing approaches, challenges, and limitations in assessing the Quality of Experience of video streaming based on network traffic. It highlights the problem of data collection in encrypted traffic that can often only be solved by active monitoring. However, this approach entails additional traffic and can be utilized if there is sufficient capacity in the network. In contrast, passive mentoring can be used to evaluate even the encrypted network traffic, but this method poses a major challenge. In the following, this section presents existing approaches of passive mentoring with underlying metrics and limitations. All of them are based on machine learning and trained on the original unencrypted network traffic. These approaches have shown some impressive results, approaching almost 100\% accuracy. However, they still have some limitations, such as being operating system or browser-specific.

The conclusion provides the summary of discussed metrics for video streaming. It once again highlights the problem of additional data traffic and mentions the passive monitoring approaches based on machine learning techniques. Finally, it raises the future question of evaluating the Quality of Experience in video streaming.


\section*{Allgemeine Bewertung}

\subsection*{Stärken der Arbeit}

This paper is an example of good scientific research that summarizes methods for evaluating the quality of experience in video streaming. The author has managed to address the research topic in an engaging and clear manner by presenting basic metrics and existing approaches. The paper is logically organized and supported by illustrations that provide a better understanding of the content. The conclusion briefly and precisely summarizes the content of the paper and raises the question of further development of the quality of experience evaluation. Finally, this work enhances the interest to further explore this topic.

\subsection*{Schwächen der Arbeit}

Potential for improvement:

\begin{itemize}
\item Figure 1 is good. However, there is some content there that may not be clear to the reader. It needs a little explanation.
\item Divide all presented approaches in chapter 3 into subchapters, which will help to distinguish these methods. Additionally, it will lead to better readability and navigation in the paper.
\item I find the question at the end very exciting. But I would go into more detail here. e.g. Shortly about future directions.
\end{itemize}  


\section*{Sachliche Korrektheit}

The paper is based on several scientific sources that can be recognized as state-of-the-art papers in the current field. The author provides correct and meaningful citations and describes the content of the sources in his own words, providing the most relevant information. Regarding the presented approaches of passive mentoring for encrypted network traffic, the author was able to briefly summarize the most important aspects, results, and shortcomings.

\section*{Äußere Form}
  
\begin{itemize}
\item No comments on spelling or grammar. The paper is written in good academic language.
\item The paper has a suitable structure. The order of the sections is logical and easy to read. However, as a suggestion for improvement, I would divide the presented methods in chapter 3 into subsections.
\item The figures are well-chosen to help the reader understand the content. All of them are referenced in the text and include self-explanatory captions.
\item The paper has a clear and well-structured appearance with suitable figures.
\end{itemize}  

\end{document}